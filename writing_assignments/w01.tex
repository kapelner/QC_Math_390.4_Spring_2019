\documentclass[12pt]{article}

\include{preamble}

\newtoggle{professormode}
\toggletrue{professormode} %STUDENTS: DELETE or COMMENT this line



\title{MATH 390.4 / 650.2 Spring 2019 \\ Writing Assignment \#1}

\author{Professor Adam Kapelner} %STUDENTS: write your name here

\iftoggle{professormode}{
\date{Due 11:59PM Thursday KY604, April 18, 2019 \\ \vspace{0.5cm} \small (this document last updated \currenttime~on \today)}
}


\begin{document}
\maketitle

\noindent Pick one of the prompts and argue for one of the choices. 

\begin{enumerate}[(a)]
\item The laws of physics are absolutely true / \\
The laws of physics are \textit{not} absolutely true / \\
The laws of physics are \textit{perhaps} absolutely true.
\item There exists an all-powerful, all-knowing and timeless being / intelligence / \\
There does \textit{not} exist an all-powerful, all-knowing and timeless being / intelligence / \\
There \textit{perhaps} exists an all-powerful, all-knowing and timeless being / intelligence.
\item Global warming is real / \\
Global warming is \textit{not} real / \\
Global warming is \textit{perhaps} real.
\item The optimal human diet for health is [insert a diet here] / \\
The optimal human diet for health is \textit{not} [insert a diet here] / \\
The optimal human diet for health is \textit{perhaps} [insert a diet here].
\item People who smoke contract lung cancer / \\
People who smoke do \textit{not} contract lung cancer / \\
People who smoke \textit{perhaps} contract lung cancer.
\end{enumerate}

\noindent It is your job to interpret the above prompts and redefine them in your own words, i.e. you must \textit{limit}, \textit{scope} and \textit{concretize} the above which are \textit{deliberately open-ended} and nebulous. To argueYou will need to formulate a model with phenomenon(s) that can be measured (explain how) and characteristics of the units under consideration. You will need to clearly define what \qu{models} are and discuss their limitations. In the appropriate context, you must write about the concepts we learned / will learn in class, including but not limited to $t ,f, g, h^*, \delta, \epsilon, e, t, z_1, \ldots, z_t, \mathbb{D}, \mathcal{H}, \mathcal{A}, n, p$, $x_{\cdot 1}, \ldots, x_{\cdot p}$, $x_{1 \cdot}, \ldots, x_{n \cdot}, \mathcal{X}, y, \mathcal{Y}$, the fidelity of the measurement process, extrapolation, interpolation, stationarity, overfitting, validation (in-sample vs. out-of-sample), model selection, machine learning, etc. \\
~\\
Specs: Your essay should be at least 10 pages double-spaced with one inch margin, 12pt Times (or Computer Modern if using \LaTeX) and be appropriately organized. Sectioning is at your preference.

\pagenumbering{gobble}

\end{document}